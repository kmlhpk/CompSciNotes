\documentclass[]{article}
\title{COMP2211 - Networks and Systems \\ Cyber Security}
\author{Kamil Hepak}
\date{2019-20}

\begin{document}
\maketitle

\begin{quote}
    \emph{``Computer security is the protection of computer systems against adversarial environments.''}
\end{quote}

We want to \textbf{allow intended} use, and \textbf{prevent unintended} use. Red vs Blue mindset - attacker vs defender.
\\
\\
Some terminology:
\begin{itemize}
    \item \textbf{Asset:} Something of value to a person or organisation.
    \item \textbf{Vulnerability:} Weakness of a system that could be accidentally or intentionally exploited to damage assets.
    \item \textbf{Threat:} Potential danger of an adversary exploiting a vulnerability.
    \item \textbf{Risk:} Asset x Threat x Vulnerability.
    \item \textbf{Adversary:} An agent that circumvents the security of a system.
    \item \textbf{Attack:} An assault on system security.
    \item \textbf{Countermeasure:} Actions/processes that an owner may take to minimize risk of a vulnerability.
    \item \textbf{Confidentiality:} Ensuring assets are only available to those who should be allowed.
    \item \textbf{Integrity:} Ensuring consistency, accuracy and trustworthiness of data.
    \item \textbf{Availability:} Ensuring that assets are always available (e.g. in the event of an attack).
    \item \textbf{Accountability:} Recording actions so that users can be held accountable for their actions.
    \item \textbf{Reliability:} Ensuring that a system can progress despite errors.
\end{itemize}

Confidentiality, integrity and availability often have to be balanced - going too far in one sector may compromise the others.

\end{document}