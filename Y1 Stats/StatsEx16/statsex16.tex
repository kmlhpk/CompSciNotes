\documentclass[]{article}

\title{\vspace{-3cm}Durham University\\
    MATH1541 Statistics \\
	Exercise Sheet 16}
\author{Kamil Hepak\\
        Tutorial Group 4}
\date{Mar 2019}

\begin{document}
\maketitle

\section{Q1}
\subsection{With Main Outliers}
Let differences be denoted $d$: $\bar{d} = 2.55$, $s_d = 4.7117$, $n_d = 15$.
\\
$H_0: \mu_d = 0$, $H_a: \mu_d \neq 0$, $\alpha = 5\%$ (two-tailed)
\\
The critical values for the $t_{n-1}$ test in this case are thus the 0.025 values from the tails of $t_{14}$ distribution. These values are $\pm 2.1448$.
\\
We calculate the test statistic, $t$, as follows:
$$t = \frac{\bar{d} - \mu_d}{\frac{s_d}{\sqrt{n_d}}}$$
$$t = \frac{2.55 - 0}{\frac{4.7117}{\sqrt{15}}}$$
$$t = 2.0961$$
\\
We can go about checking the significance of the test statistic in one of many ways, but the most straightforward is to compare it to the critical value. Since $t < 2.1448$, we fail to reject the null hypothesis at the 5\% significance level - the evidence does not suggest $\mu_d \neq 0$.

\subsection{Assumptions}
To make this a valid test, we must assume:
\begin{itemize}
    \item Each observation is independent
    \item Each observation is selected with a simple random sample
    \item The original distribution of the plant heights is reasonably Normal (incl. no outliers and little to no skew)
\end{itemize}


\subsection{Without Main Outliers}
There are three outliers in the data set - two are contained in the cross-fertilised portion, while one is in the self-fertilised. The two cross-fertilised outliers are further from the lower outlier limit (17.65625, calculated as $Q_2 - 1.5\cdot$IQR) for the cross-fertilised data than the other outlier is from the self-fertilised limit, as well as being absolutely lower than the other outlier. Therefore, they and their corresponding self-fertilised data points have been omitted. Additionally, when converted to differences, the only outliers are the ones stemming from the cross-fertilised data, further reinforcing the decision to omit those two data points.
\\
\\
Let differences be denoted $d$: $\bar{d} = 4.0481$, $s_d = 2.7260$, $n_d = 13$.
\\
$H_0: \mu_d = 0$, $H_a: \mu_d \neq 0$, $\alpha = 5\%$ (two-tailed)
\\
The critical values for the $t_{n-1}$ test in this case are thus the 0.025 values from the tails of $t_{12}$ distribution. These values are $\pm 2.1788$.
\\
We calculate the test statistic, $t$, as follows:
$$t = \frac{\bar{d} - \mu_d}{\frac{s_d}{\sqrt{n_d}}}$$
$$t = \frac{4.0481 - 0}{\frac{2.7260}{\sqrt{13}}}$$
$$t = 5.3542$$
\\
We can go about checking the significance of the test statistic in one of many ways, but the most straightforward is to compare it to the critical value. Since $t > 2.1788$, we reject the null hypothesis at the 5\% significance level - the evidence does not suggest $\mu_d = 0$.

\end{document}