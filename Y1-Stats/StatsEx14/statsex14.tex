\documentclass[]{article}

\title{\vspace{-3cm}Durham University\\
    MATH1541 Statistics \\
	Exercise Sheet 14}
\author{Kamil Hepak\\
        Tutorial Group 4}
\date{Feb 2019}

\begin{document}
\maketitle
\section{Q1}
$\bar{x} = \frac{184.7}{17} = 10.8647$
\\
$s_x = \sqrt{\frac{2404.41-17\times 10.8647}{16}} = 4.9849$
\\
For $t_{n-1}$, ie. $t_{16}$, $\mu \in [10.8647 \pm 2.120\cdot\frac{4.9849}{\sqrt{17}}]$
\\
$\mu \in [8.30, 13.43]$
\\
$\sigma$ is unknown and $n$ is fairly small - assume underlying data is Normally distributed.

\section{Q5}
$\bar{x} = \frac{51.6}{6} = 8.6$
\\
$\sigma^2 = 0.4$
\\
Since $\sigma$ is known but $n$ is small, assuming candle lifetimes are distributed Normally: $\mu \in [8.6 \pm 2.3263\cdot\frac{\sqrt{0.4}}{\sqrt{6}}]$
\\
$\mu \in [8.00, 9.20]$

\section{Q6}
$\bar{x} = 1.90$
\\
$s_x = 0.66$
\\
$\mu \in [1.90 \pm 2.8982\cdot\frac{0.66}{\sqrt{18}}]$
\\
$\mu \in [1.45, 2.35]$
\\
$\sigma$ is unknown but $n$ is fairly large - assume underlying data is Normally distributed. This can be checked with a box plot and/or normal quantile plot.

\section{Q7}
$\bar{x} = 22.57$
\\
$s_x = 1.07$
\\
$\mu \in [22.57 \pm 1.1503\cdot\frac{1.07}{\sqrt{100}}]$
\\
$\mu \in [22.45, 22.69]$


\end{document}
