\documentclass[]{article}

\title{\vspace{-3.0cm}Durham University\\
    MATH1541 Statistics \\
	Exercise Sheet 12}
\author{Kamil Hepak\\
        Tutorial Group 4}
\date{Feb 2019}

\begin{document}
\maketitle

\section{Q1}
\subsection{a)}
$c$ = 2.120
\\
Under $\sim N$, $c$ = 1.960

\subsection{b)}
$c$ = 1.895
\\
Under $\sim N$, $c$ = 1.645

\subsection{c)}
$t_{30}$, $c$ = 2.042 
\\
$t_{40}$, $c$ = 2.021 
\\
$t_{35}$, $c \approx \frac{2.042+2.021}{2}$ = 2.0315 (Calculator gives 2.0301)
\\
Under $\sim N$, $c$ = 1.960

\subsection{d)}
P($T > 1.5) \approx$ 0.080
\\
Under $\sim N$, P($T > 1.5) \approx$ 0.067

\subsection{e)}
P($T > 1.5 \cap T < -1.5) \approx$ 0.16
\\
Under $\sim N$, P($T > 1.5) \approx$ 0.134
\newpage

\section{Q3}

\subsection{a)}
$\bar{X}$ will have an approximately Normal distribution - that is to say, $\bar{X} \sim N(\mu,\frac{\sigma^2}{n})$. Because $X$ has a distribution with mean $\mu$ and variance $\sigma^2$, and $n \geq 10$, we can use the Central Limit Theorem to assume $\bar{X}$'s distribution.

\subsection{b)}
When the underlying distribution of $X$ is Normal.

\subsection{c)}
Assuming $\sim N(0,1)$, since $\sigma$ is known, $c=1.960$

\subsection{d)}
\subsubsection{i)}
Plot attached.
\\
The normal quantile plot is not very linear - an incredibly ``fat pen'' would be necessary to encapsulate the data. The use of a $t$-distribution would probably not be fully appropriate in this scenario.
\subsubsection{ii)}
$t_{12}$, $c = 2.179$
\subsubsection{iii)}
$t_{12}$, $c = 0.695$

\end{document}