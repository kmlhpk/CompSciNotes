\documentclass[]{article}
\title{\vspace{-2.0cm}Durham University\\
    MATH1541 Statistics \\
	Exercise Sheet 8}
\author{Kamil Hepak\\
        Tutorial Group 4}
\date{Dec 2018}

\begin{document}
\maketitle

\section{Find $s_{\epsilon}$.}
To find $s_{\epsilon}$, we use the formula $s_{\epsilon} = s_{y}\sqrt{1 - R^2}$. As per the R output, $R^2 = 0.925$ and $s_{y} = 4099.8$. Thus, $s_{\epsilon} = 1122.776$.

\section{Assess the relative value of the predictors.}
\begin{displaymath}
    \begin{array}{c|cccccc}
        Variable & x_1 & x_2 & x_3 & x_4 & x_5 & x_6\\
        \hline
        |\hat{b}_i|s_i & 39.65 & 1363.50 & 5127.92 & 535.86 & 9636.55 & 14114.14
        \end{array}    
\end{displaymath}\\
Using the relationship ``value of variable $x_i \propto |\hat{b}_i|s_i$'', we can see that Site 6 is the greatest contributor, and thus the most relevant predictor, by a fair margin. Sites 5 and 3 also contribute a large amount to the value of the prediction, Sites 2 and 4 contribute considerably less, and Site 1 contributes such an insignificant change that it may be a candidate for exclusion when computing $\hat{y}$. All 6 of the variables' standard deviations are sufficiently similar, thus we can fairly confidently compare them directly. Before excluding any variable, we would need access to data about the residuals for $y$ - if their plots against any variable exhibited heteroscedasticity, then we may consider excluding that variable.

\section{Predict the value of run-off volume, and find a 90\% confidence interval.}
To predict $y$, we use the formula:
$$\hat{y} = \hat{a} + \hat{b}_1\cdot x_1 + \hat{b}_2\cdot x_2 + \hat{b}_3\cdot x_3 + \hat{b}_4\cdot x_4 + \hat{b}_5\cdot x_5 + \hat{b}_6\cdot x_6$$
Following the R output and given values of $x_{1-6}$, the calculation is $\hat{y} = -12.8(7) - 664.4(4) + 2270.7(4) + 69.7(10) + 1916.5(10) + 2211.6(12) = 52736.8$.
Assuming that, for these specific values of $x_{1-6}$, $y\sim N(52736.8, 1122.776^2)$, we can compute the 90\% confidence interval for $y$ as follows: $52736.8 \pm 1.6449(1122.776)$. This results in an interval of (50889.9, 54583.7).

\end{document}