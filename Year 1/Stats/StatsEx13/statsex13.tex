\documentclass[]{article}

\title{\vspace{-3.0cm}Durham University\\
    MATH1541 Statistics \\
	Exercise Sheet 13}
\author{Kamil Hepak\\
        Tutorial Group 4}
\date{Feb 2019}

\begin{document}
\maketitle

\subsection{Q1}
\subsubsection{a)}
The probability of a student passing the exam is not constant, as a student's understanding is dependent on many other factors, thus a binomial distribution is not appropriate in this case.
\subsubsection{b)}
Again the probability of getting a correct answer is not constant or independent (later questions are dependent on good understanding of earlier material), thus a binomial distribution will not give a good answer.
\subsection{c)}
The differing temperatures likely affect the probability of material failure, and the fact the same material is being used for this test means the experiment is independent - once again, a binomial distribution will not be conducive to correct calculations.

\subsection{Q4}
\subsubsection{a)}
\subsubsection{b)}
\subsubsection{c)}

\subsection{Q5}
\subsubsection{a)}
If we assume the probability, 0.4, of a person being a Conservative voter, and that each person's probability of voting that way is independent, then we also have a binary choice (Tory voter, or not), thus allowing us to model $Y$ as binomial.
\subsubsection{b)}
\begin{tabular}
    
    
\end{tabular}

\subsubsection{c)}
\subsubsection{d)}
\subsubsection{e)}
\subsubsection{f)}
\subsubsection{g)}
\subsubsection{h)}


\end{document}