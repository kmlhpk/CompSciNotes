\documentclass[]{article}

\title{\vspace{-3.0cm}Durham University\\
    MATH1541 Statistics \\
	Exercise Sheet 13}
\author{Kamil Hepak\\
        Tutorial Group 4}
\date{Feb 2019}

\begin{document}
\maketitle

\subsection{Q1}
\subsubsection{a)}
The probability of a student passing the exam is not constant, as a student's understanding is dependent on many other factors, thus a binomial distribution is not appropriate in this case.
\subsubsection{b)}
Again the probability of getting a correct answer is not constant or independent (later questions are dependent on good understanding of earlier material), thus a binomial distribution will not give a good answer.
\subsection{c)}
The differing temperatures likely affect the probability of material failure, and the fact the same material is being used for this test means the experiment is independent - once again, a binomial distribution will not be conducive to correct calculations.

\subsection{Q4}
\subsubsection{a)}
\subsubsection{b)}
\subsubsection{c)}

\subsection{Q5}
\subsubsection{a)}
If we assume the probability, 0.4, of a person being a Conservative voter, and that each person's probability of voting that way is independent, then we also have a binary choice (Tory voter, or not), thus allowing us to model $Y$ as binomial.
\subsubsection{b)}
\begin{tabular}{c|c|c|c|c|c|c|c}
    $y$&0&1&2&3&4&5&6\\
    \hline
    \\[\dimexpr-\normalbaselineskip+4pt]
    P($Y=y$)&0.04666&0.1866&0.3110&0.2765&0.1382&0.03686&4.096$\times10^{-3}$
\end{tabular}
\subsubsection{c)}
E($Y$) = $\sum_{i=0}^{6} y \cdot \textrm{P}(Y=y)$
\\
E($Y$) = 2.3998 
\\
Var($Y$) = $\left(\sum_{i=0}^{6} y^2 \cdot \textrm{P}(Y=y)\right) - \textrm{E}(Y)^2$
\\
Var($Y$) = 7.1993 - 2.3998$^2$ = 1.4402
\subsubsection{d)}
$np$ = 2.4
\\
$np(1-p)$ = 1.44
\subsubsection{e)}
$Y \sim \textrm{B}(900,0.4)$
\\
$Z \sim \textrm{N}(360,(6\sqrt{6}^2))$
\\
P($333 \leq Y \leq 378$) $\approx$ P($332.5 \leq Z \leq 378.5$)
\\
P = 0.8653
\subsubsection{f)}
$P = \frac{Y}{900}$
\\
Assume $P \sim \textrm{N}(0.4, 2.67\times 10^{-4})$
\\
\subsubsection{g)}
\subsubsection{h)}



\end{document}