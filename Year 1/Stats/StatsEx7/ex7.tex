\documentclass[]{article}
\title{Durham University\\
    MATH1541 Statistics \\
	Exercise Sheet 7}
\author{Kamil Hepak\\
        Tutorial Group 4}
\date{Nov 2018}

\begin{document}
\maketitle

\section{Estimate a and b.}
We are given that the relationship between $y$ and $t$ is thought to be of a hyperbolic form, namely following the equation $y = \frac{at}{b+t}$. Initially, I tried rearranging the formula thus:
$$\frac{1}{y}=\frac{b+t}{at}$$
$$\frac{1}{y}=\frac{b}{at}+\frac{t}{at}$$
$$\frac{1}{y}=\frac{b}{a}\frac{1}{t}+\frac{1}{a}$$
Plotting the reciprocal of $y$ against the reciprocal of $t$ produces a linear-seeming plot, but the residual plot for this transformation does not do huge amounts to suggest homoscedasticity. A different arrangement of the equation produces an even better result:
$$\frac{1}{y}=\frac{b+t}{at}$$
$$\frac{t}{y}=\frac{b}{a}+\frac{t}{a}$$
$$\frac{t}{y}=\frac{1}{a}t+\frac{b}{a}$$
As visible on the attached plots, the relationship between $\frac{t}{y}$ and $t$ is very linear and homoscedastic. In this form, we have an equation that resembles one of linear regression - estimating $\frac{t}{y}$, the intercept is $\frac{b}{a}$ and the coefficient is $\frac{1}{a}$.
From the R calculations, we can see that the linear model function returns a value of 0.3451 for $\frac{1}{a}$, and a value of 1.8800 for $\frac{b}{a}$. Rearranging, we estimate $a$ to be 2.8977 and $b$ to be 5.4477.

\section{Predict water uptake at time $t$ = 17.}
Let $q = \frac{t}{y}$. Thus, $\hat{q} = 1.88 + 0.3451t$. Substituting $t = 17$, we find $\hat{q} = 7.7467$. Rearranging to find the water uptake estimate $\hat{y}$, we take $\frac{17}{\hat{q}}$ to obtain a result of $\hat{y} = 2.1945$.

\section{Find an interval which should contain 95\% of the values of the response variable at time $t = 17$.}
First, we find our rmsr $s_{\epsilon}$ by using the formula $s_{\epsilon} = s_{y}\sqrt{1 - r^2}$. With $s_{y} = 2.522$ and $r = 0.9866$, we find $s_{\epsilon} = 0.4109$.
\\
Let $Q$ be the random variable representing the transformed values of the uptakes at time $t = 17$. Assuming $Q\sim N(7.7467, 0.4109^2)$ we can find the 95\% confidence interval for $Q$ by using the equation $7.7467 \pm 1.96(0.4109)$. This results in an interval in terms of $q$ of (6.9413, 8.5521) - rearranging for $y$, we get an interval of (1.9878, 2.4491).

\section{Check any assumptions necessary in making your predictions.}

\subsection{Linearity of Model - Accuracy of Hyperbolic Law Prediction}
A linear, homoscedastic relationship is crucial for making predictions using regression. As visible on the plot of $\frac{t}{y}$ against $t$, the transformed data follows an almost too-good-to-be-true linear relationship. An $r$ of 0.9866 suggests an incredibly high positive correlation, and the regression line fits neatly through the data points. This suggests that a hyperbolic relationship between the response and explanatory variable is fitting.

\subsection{Homoscedasticity}
As visible on the plot of the residuals of $\frac{t}{y}$ against $t$, there appears to be no correlation whatsoever between the two variables, strongly implying homoscedasticity.

\subsection{Normal Distribution of Residuals}
To accurately predict a confidence interval for a certain value of the explanatory variable, one must assume the residuals are normally distributed. A normal quantile plot of the residuals of $\frac{t}{y}$ confirms this - the data is very linear, implying normal distribution.

\subsection{Assumptions that I cannot check}
Of course, all of these predictions rely on the assumption that change in $t$ causes change in $y$, and that there isn't a confounding variable linking the two. We also assume that the sample we are provided with is representative of the population enough  to predict from.

\end{document}