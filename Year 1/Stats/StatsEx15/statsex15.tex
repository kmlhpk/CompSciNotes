\documentclass[]{article}

\title{\vspace{-3cm}Durham University\\
    MATH1541 Statistics \\
	Exercise Sheet 15}
\author{Kamil Hepak\\
        Tutorial Group 4}
\date{Mar 2019}

\begin{document}
\maketitle

\section{Q1}
\subsection{a) b)}
$H_0: \mu = 8.0$, $H_a: \mu \neq 8.0$
\\
$p = \frac{8.6-8}{\frac{\sqrt{0.4}}{\sqrt{6}}} = 2.3237$
\\
$P$ = 0.9797
\\
$\alpha = 5\%$, CV = $\pm 1.9600$, thus reject $H_0$
\\
$\alpha = 1\%$, CV = $\pm 2.5758$, thus fail to reject $H_0$

\subsection{c)}
$X \sim \textrm{Bin}(1000, 0.01)$
\\
E$(X)$ = 10, Var($X$) = 9.9
\\
Therefore, over 1000 experiments, we would expect 10 to give us a Type I error.

\section{Q2}
\subsection{a)}
$H_0: \mu = 2.0$, $H_a: \mu \neq 2.0$
\\
$\alpha = 10\%$, $t_{17}$ value = 1.7396
\\ 
CI: $\mu \in [1.9 \pm 1.7396\cdot \left( \frac{0.66}{\sqrt{18}} \right)]$
\\
$2 \in [1.63, 2.17]$, thus fail to reject $H_0$

\subsection{b)}
The $t$-tables do not provide a value, at $t_{17}$, for the $p$-value of -0.6428.

\section{Q7}
\subsection{a)}
False - as per section 7.8 of the lecture notes, hypothesis tests should not be carried out on data that suggests a hypothesis (``many interesting, possibly significant, findings'').

\subsection{b)}
False - as per section 7.5 of the lecture notes, when $\sigma$ is unknown but the sample is large, any sampling distribution will be appropriate for use with a Normal-based test.

\subsection{c)}
True - as per section 7.5 of the lecture notes, when $\sigma$ is unknown and $n$ is small, a Normal sampling distribution is required to validate the use of the $t$ distribution in hypothesis testing.

\subsection{d)}
False - a CI is not a random interval, therefore saying $\mu$ has a probability is nonsensical.

\subsection{e)}
True - the number of type I errors in $n$ independent experiments where we carry out a
hypothesis test at a 1\% level of significance is distributed Bin($n$, 0.01).

\subsection{f)}
False - 0.01 is the probability of a Type I error; that is, rejecting $H_0$ when it is actually true.

\subsection{g)}
True - this is how one performs a hypothesis test using the CI method.

\subsection{h)}
False - to hypothesis test at significance level $\alpha$\%, one must construct the $1-\alpha$\% CI. Additionally, if the test statistic falls outside a CI, one would reject $H_0$.

\subsection{i)}
True, as per section 7.7 of the lecture notes; the same reasoning as for part e).

\end{document}